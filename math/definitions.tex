% ---------------------------------------- %
\documentclass[12pt]{article}
\usepackage{amsmath}
\usepackage{graphicx}
\usepackage{subcaption}
\usepackage{float}
\usepackage{hyperref}
\usepackage{amssymb}
\usepackage{amsthm}
% Add the package for institution %
\usepackage{authblk}
% Add the package for bibliography %
\usepackage{biblatex}
\usepackage{amsmath}
\usepackage{amssymb}
\usepackage{amsthm}
\usepackage{mdwlist}
%\usepackage{amsrefs}
\usepackage{dsfont}
\usepackage{mathrsfs}
\usepackage{stmaryrd}
\usepackage[all]{xy}
\usepackage[mathcal]{eucal}
\usepackage{verbatim}  %%includes comment environment
\usepackage{fullpage}  %%smaller margins
\usepackage{hyperref}


\newtheorem{theorem}{Theorem}
\newtheorem{proposition}[theorem]{Proposition}
\newtheorem{lemma}[theorem]{Lemma}
\newtheorem{corollary}[theorem]{Corollary}
\newtheorem*{defi}{Definition}
\theoremstyle{definition}
\newtheorem{definition}[theorem]{Definition}
\newtheorem{nondefinition}[theorem]{Non-Definition}
\newtheorem{exercise}[theorem]{Exercise}

\title{Equivariant Neural Networks for chemistry.}
\author{Caridad, Rodrigo}
% let me add institution %
\affil{University of Chicago}

\date{\today}

\begin{document}
\maketitle

In this document we define the main mathematical concepts, and equations behind all the models that are described in the repo.

\section{MLPs}

\section{Invariance and Equivariance}

For simplicity we will consider the \(n\)-dimensional euclidean group \(\text{E}(n)\). Given a group element \(\epsilon \in \text{E}(n)\) we define its representation as the map \(\rho : \text{E}(n) \to \mathbb{R} ^ {n \times n}  \), where \(\rho(\epsilon)\) is a matrix.

\begin{definition}
    We say a function \(f : \mathbb{R}^n \to \mathbb{R}^n\) is \(\text{E}(n)\) invariant if for all \(x \in \mathbb{R}^n\) and all \(\epsilon \in \text{E}(n)\) we have that \(f(\rho(\epsilon)x) = f(x)\).
\end{definition}

Or in other words, The fuction maps all euclidean transformations of a given input to the same value.

\begin{definition}
    We say a function \(f : \mathbb{R}^n \to \mathbb{R}^n\) is \(\text{E}(n)\) equivariant if for all \(x \in \mathbb{R}^n\) and all \(\epsilon \in \text{E}(n)\) we have that \(f(\rho(\epsilon)x) = \rho(\epsilon)f(x)\).
\end{definition}

Which means the function maps a transformed value to another value transformed in the same way.

\section{GNNs}

\section{Equivariant GNNs}

\end{document}